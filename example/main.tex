\documentclass[a4paper,11pt]{article}
\usepackage{preamble}

\title{Tutorial 3}

\author{Computer Security \\
Orfeas Stefanos Thyfronitis Litos \\
School of Informatics \\
University of Edinburgh}

\date{}

\begin{document}
  \maketitle

  \epigraph{Don't roll your own crypto, bro.}{Joseph Cox, VICE.com}

  The tutorial is formative. This means that you will not submit your work and
  it will not contribute in any way towards your final mark for the course. We
  do however provide some challenges so that you can self-assess your
  understanding of the material. Exercises marked with ``*'' are a bit more
  challenging; they also form a sequence, so you'll need the solution of the
  previous starred exercise to move on to the next.

  In this tutorial, we will discuss the needed theory (keeping it to a
  minimum)\footnote{We will throw various terms around. We will discuss only the
  most relevant ones, but you are welcome to look up the rest if you're
  curious.} and gain some hands-on experience with various cryptographic
  primitives by exploring the
  \texttt{pycryptodome}\footnote{\url{https://www.pycryptodome.org}}\footnote{In
  fact we use \texttt{pycryptodomex}, because \texttt{pycryptodome} uses the
  same module name as \texttt{pycrypto}, an outdated crypto library that we
  unfortunately can't uninstall from DICE.} library provided for Python 3. We
  will also try out
  \texttt{hashlib}\footnote{\url{https://github.com/python/cpython/blob/3.8/Lib/hashlib.py}}.
  % TODO: add examples with
  % \texttt{pyaes}\footnote{\url{https://github.com/ricmoo/pyaes}}
  % and update README.md when done
  All these libraries are available on DICE machines, so don't just read but try
  out the code yourself!

  There are many more crypto libraries out there, implemented in and available
  for every (useful) programming language. You can choose whichever suits you
  best depending on your usecase, the choice we made here is quite arbitrary.
  But please don't ever implement your own crypto library and proceed to use it
  in production code -- you'll probably do a worse job than experts both in
  terms of speed and (more crucially) security\footnote{Writing a toy
  implementation of a cryptographic algorithm/protocol on the other hand may be
  an educational and fun experience. It may even be the first step for you to
  become a professional crypto coder!}.

  \section{Cryptographic Hash functions}
  \subsection{Theory}
    Just as a brush up, a hash function $\mathcal{H} : \{0, 1\}^* \rightarrow
    \{0, 1\}^n$ compresses\footnote{or expands, it if its length is less than
    $n$} an input of arbitrary length down to $n$ bits. Two prominent security
    properties desired by a cryptographic hash function are:
    \begin{itemize}
      \item One-wayness: Given $y \in \{0, 1\}^n$, it is hard to find $x \in
      \{0, 1\}^*$ such that $\mathcal{H}(x) = y$.
      \item Collision-freeness: It is hard to find $x_1 \neq x_2 \in \{0, 1\}^*$
      such that $\mathcal{H}(x_1) = \mathcal{H}(x_2)$.
    \end{itemize}

    The two most common hash functions are SHA-2 and SHA-3, which are
    standardized by NIST. Each comes in four variants, depending on their output
    size (224, 256, 384, and 512 bits). If you are free to choose in your
    project, opt for SHA3 as it is newer and shinier. Here we'll stick to the
    256 versions because:
    \begin{enumerate}
      \item they are big enough to be secure for most applications
      \item they give a nice round-number\footnote{\url{https://xkcd.com/1000/}}
      output of 256 bits (a.k.a. 32 bytes).
    \end{enumerate}

    For historical reasons\footnote{SHA-1 had only one output size, 160 bits}
    when we refer to a particular SHA-2 variant we drop the ``2''. Therefore the
    256 variant of SHA-2 is written as SHA-256. The ``3'' of SHA-3 on the other
    hand is always kept, e.g. the 384 variant of SHA-3 is written as SHA3-384.
    Confusing, I know.

  \subsection{Practice}
    Enough talk, let's try to hash ``\texttt{Hello world!}'' with SHA256!

    \getenv[\hellosha]{HELLO_SHA}
    \codelisting{\hellosha}
    If you're anything like me, this should happen when you run the above:
    \outputlisting{\hellosha}

    We used the \texttt{new()} method of the \texttt{SHA256} class, passing it
    the string \texttt{Hello world!} in \textit{binary} (notice the \texttt{b}
    before the string) and then called the method
    \texttt{hexdigest()}\footnote{\textit{digest} is just an old-fashioned way
    of saying \textit{hash}} of the object returned by \texttt{SHA256.new()},
    which gave us a printable string with the desired hash.

    This is essentially the same as the following:

    \getenv[\helloshahex]{HELLO_SHA_HEX}
    \codelisting{\helloshahex}
    \outputlisting{\helloshahex}
    The only difference here is that we use the builtin python method
    \texttt{hex()} on the binary result of \texttt{SHA256.new(b"Hello
    world!").digest()} to get the hash in printable form. It's good practice to
    use the binary format returned by \texttt{digest()} for pretty much
    everything else apart from printing.

    If we want to hash a string that is the concatenation of various parts, we
    can use \texttt{update()}:

    \getenv[\helloshaupdate]{HELLO_SHA_UPDATE}
    \codelisting{\helloshaupdate}
    \outputlisting{\helloshaupdate}
    This avoids the manual concatenation and, because of how most hash functions
    work, can result in slightly more efficient code.

    Just as a sanity check, let's hash ``\texttt{Hello world!}'' using
    \texttt{hashlib} instead of \texttt{pycryptodome} and ensure that the output
    is the same.

    \getenv[\helloshahashlib]{HELLO_SHA_HASHLIB}
    \codelisting{\helloshahashlib}
    \outputlisting{\helloshahashlib}
    Phew! Works as expected. It turns out that the syntax is slightly different
    (spotting the differences is left as an exercise for the reader), but in
    effect both APIs give us exactly the same methods to play with. Don't ask me
    who is the copycat.

    \begin{exercise}
      \label{ex:hash:name}
      Write a script that prints the SHA256 hash of your full name.
    \end{exercise}

    \solution{SHA_NAME}

    \begin{exercise}
      \label{ex:hash:sourcecode}
      Check that the .iso of alpine Linux
      (\url{https://alpinelinux.org/downloads/}) ``mini root filesystem'' for
      armv7\footnote{\url{http://dl-cdn.alpinelinux.org/alpine/v3.10/releases/armv7/alpine-minirootfs-3.10.3-armv7.tar.gz}}
      hashes to the value claimed on the
      webpage\footnote{\url{http://dl-cdn.alpinelinux.org/alpine/v3.10/releases/armv7/alpine-minirootfs-3.10.3-armv7.tar.gz.sha256}}.
      \emph{Hint:} You can download the .iso and read the file with
      \texttt{open()} or use \texttt{urlopen} from \texttt{urllib.request} to
      download it directly from the script.
    \end{exercise}

    \solution{SHA_SOURCE_CODE}

    \begin{exercise*}
      \label{ex:hash:bruteforce}
      Find the smallest bitstring such that the hex representation of its SHA256
      hash starts with 0123. \emph{Hint:} Find a way to exhaustively test all
      bitstrings, from smaller to larger, until you find a match. A useful data
      type is \texttt{bytearray}.
    \end{exercise*}

    \solution{BRUTEFORCE}

  \section{Symmetric key encryption}
  \subsection{Theory}
    Let's now do a quick refresher on what we've learned on symmetric key
    encryption schemes. Such a scheme consists of a pair of functions, one for
    encryption (takes a key and a plaintext, gives a ciphertext) and one for
    decryption (takes a key and a ciphertext, gives a plaintext):
    \begin{align*}
      \mathit{Enc}: \{0, 1\}^{\lambda} \times \{0, 1\}^* \rightarrow \{0, 1\}^*
      & \\
      \mathit{Dec}: \{0, 1\}^{\lambda} \times \{0, 1\}^* \rightarrow \{0, 1\}^*
      & \enspace.
    \end{align*}
    For such a scheme to work as expected, we require that if an encrypted
    message is then decrypted with the same key, we should get back the original
    message:
    \begin{equation*}
      \emph{Correctness:} \: \forall k \in \{0, 1\}^{\lambda}, \forall m \in
      \{0, 1\}^*: \mathit{Dec}(k, \mathit{Enc}(k, m)) = m \enspace.
    \end{equation*}

    Unfortunately that's not enough because it doesn't say anything regarding
    security\footnote{Observe that the useless scheme $\mathit{Enc}(k, x) =
    \mathit{Dec}(k, x) = x$ satisfies correctness.}. In particular we'd want
    some notion of ``the ciphertext leaks no bits of the plaintext''. The formal
    version of this notion is \emph{semantic security}, which is exactly the
    kind of security you should expect from any respectable symmetric key
    encryption scheme.

    A little bit more unavoidable theory before the code: The most widely used
    building block for symmetric key encryption schemes are \emph{block
    ciphers}\footnote{The reason: they are generally faster than the
    alternative, \emph{stream ciphers}.}. A block cipher looks mostly like an
    encryption scheme but takes as plaintext (or ciphertext) a small,
    fixed-length message, which means that it's not suitable for encrypting my
    latest 3-page-long love letter.

    The simplest solution is to split up the plaintext into chunks of the
    correct size (padding the last one if needed) and encrypt each chunk
    separately, then do the reverse for decryption. That's also a bad idea. If
    my letter contains the phrase ``I love you Alice'' many times (and the
    repetitions happen to be aligned with the block size), an attacker that sees
    the ciphertext can easily deduce that I use the same phrase over and
    over\footnote{If that doesn't seem scary enough, take a look at the images
    in \url{https://en.wikipedia.org/wiki/Block_cipher_mode_of_operation#ECB}.}.
    That's not semantically secure.

    To get security back, we have to somehow make each block depend on the
    rest\footnote{Therefore achieving \emph{diffusion}.}. There is an abundance
    of ways to achieve this dependence. These ways are called \emph{modes of
    operation}, each with its own pros and cons\footnote{Who would have guessed
    that cryptography is complicated...}. The naive way described above is the
    Electronic Codebook (ECB) and we never use it in
    practice\footnote{\url{https://www.reddit.com/r/ProgrammerHumor/comments/6m6bvv/all_block_cipher_modes_are_beautiful/}}.
    The mode we care about in this tutorial is Cipher Feedback (CFB). We will
    see the most crucial details as we go during the practical part (which we
    finally reached!), but a full-blown comparison of all modes is beyond the
    aims of this tutorial\footnote{In practice, we often need
    \emph{authenticated encryption}. We then use the GCM mode of operation.}.

  \subsection{Practice}
    The by far most commonly used block cipher is AES. In fact it's so common
    that it has been implemented in hardware and relevant instructions are
    widely available on commercial CPUs\footnote{Which makes the following
    phrase completely valid: ``Our CPU boasts of an instruction set enabling a
    range of commonly used operations, such as bitshifts, logical AND, primary
    school arithmetic and polynomial multiplication over the Galois Field
    $2^8$.''}, so it is blazingly fast. We will therefore use this cipher for
    our tutorial. The block size of AES is 128 bits (a.k.a. 16 bytes) and it
    needs a key of length 128, 192, or 256 bits (i.e. 16, 24, or 32 bytes). Just
    to be on the safe side, I recommend using 192 bits, because it is much
    faster and in practice as secure as 256 bits\footnote{Different key sizes
    also result in slightly different variations of the algorithm, but the API
    is exactly the same.}.

    Let's see the most basic example: Encrypting a phrase with a key and then
    decrypting back to the original. We will use the CFB mode for this. This
    mode needs some good quality randomness for the
    IV\footnote{\emph{initialization vector}, a random number needed by the CFB
    mode}, which we will get from the \texttt{Random} module of
    \texttt{pycryptodome} (alternatively we could have used the \texttt{secrets}
    module). A general piece of advice: Never, ever, ever use randomness that is
    not dubbed ``cryptographically secure'' for your cryptographic tasks. For
    example, the \texttt{random} Python module gives very predictable, and thus
    bad quality, randomness and so we stay away from it. To be honest, in
    production I'd rather use a library that handles the IV internally and does
    not expose it at all to protect myself from footguns.

    \getenv[\aesbasic]{AES_BASIC}
    \codelisting{\aesbasic}
    This is the expected output:
    \outputlisting{\aesbasic}

    Observe that we had to construct two \texttt{AES} objects, one for
    encryption and one for decryption. This is because all modes of operation
    (apart from ECB) encrypt the same message to different ciphertexts depending
    on how many bytes have been encrypted before during this run of operations.
    To be decrypted properly, each ciphertext must be processed by an
    \texttt{AES} object that has already decrypted as many bytes as were
    encrypted before the encryption of the ciphertext (and of course with the
    same IV). Let's see a clarifying example:

    \getenv[\aesmode]{AES_MODE}
    \codelisting{\aesmode}

    \begin{exercise}
      \label{ex:aes:name}
      Encrypt your name with AES-192-CFB under a random key and IV and
      successfully decrypt the resulting ciphertext.
    \end{exercise}

    \solution{AES_NAME}

    \getenv[\classifiedkey]{CLASSIFIED_KEY}
    \getenv[\classifiediv]{CLASSIFIED_IV}
    \getenv[\classifiedciphertext]{CLASSIFIED_CIPHERTEXT}
    \begin{exercise}
      \label{ex:aes:classified}
      Decrypt the ciphertext \topythonbytes{\classifiedciphertext} which has
      been encrypted with AES-256-CFB, under key \topythonbytes{\classifiedkey}
      and IV \topythonbytes{\classifiediv}. Ensure the plaintext is in English.
    \end{exercise}

    \solution{AES_CLASSIFIED}

    \getenv[\ctrnonce]{CTR_NONCE}
    \getenv[\ctrciphertext]{CTR_CIPHERTEXT}
    \begin{exercise*}
      \label{ex:aes:counter}
      Decrypt the ciphertext \topythonbytes{\ctrciphertext} which has been
      encrypted with AES-192-CTR (notice the different mode!\footnote{Refer to
      the documentation at
      \url{https://pycryptodome.readthedocs.io/en/latest/src/cipher/aes.html#Crypto.Cipher.AES.new}
      to find out how it works.}). The key is the concatenation of the answer to
      exercise~\ref{ex:hash:bruteforce} with itself 12 times. The \emph{nonce}
      is \texttt{\ctrnonce}. The AES object has encrypted
      \topythonbytes{0123456789} prior to encrypting our plaintext. Ensure that
      the resulting plaintext starts with \texttt{-----BEGIN RSA PRIVATE
      KEY-----} and ends with \texttt{-----END RSA PRIVATE KEY-----}.
      \textit{Hint:} do not touch the \textit{counter}, just let the tool use
      the default one.
    \end{exercise*}

    \solution{AES_CTR}

  \section{Public key cryptography}
    This is the last section of this Tutorial, so let's make it worth it! We'll
    see why public key cryptosystems are useful and will try out PKCS1 OAEP for
    encryption and PKCS1 PSS for digital signatures, both provided by
    \texttt{pycryptodome}. Since these tools are more complex than the previous
    ones under the hood, we will gloss over more theoretical details than before
    but we won't throw away the essence.

  \subsection{Theory}
    Symmetric key encryption schemes have some shortcomings. In particular,
    each pair of communicating parties need a unique, securely stored and used
    key. This means that for a clique of $N$ parties to communicate, each
    party has to handle securely $N-1$ keys and be ready to update them in
    case the counterparty mismanages it. The total count of keys will be
    $N(N-1)$.

    The problem above is ``just'' logistics. There are several more
    fundamental problems however. Here we state just a few: Firstly, securely
    communicating the key itself when first establishing connection is
    impossible, except for in-person key generation and sharing\footnote{Such
    primitive methods are obviously shunned upon.}. This is because presumably
    there is no secure channel through which two parties can privately share
    the key to begin with. (If such a channel existed, parties could simply
    use this for their communication.) Secondly, since there is no existing
    secure channel, how can one party trust that the other party is indeed the
    intended one? Packets on the Internet pass through innumberable switches
    and routers -- if their contents are in plaintext, a malicious router can
    just pretend it is the intended recipient and respond as it wishes. This
    is called a \emph{Man in the Middle} attack and it cannot be mitigated
    through purely symmetric encryption. Thirdly, since each key is shared by
    two parties, there is no way for Alice to prove to a Judge that Bob told
    her something if the Judge doesn't trust neither Alice nor Bob. This is
    called the issue of \emph{repudiation} and to solve it we again need
    asymmetric schemes.

    The two primary tools are \emph{public key encryption} and
    \emph{digital signatures}. In both cases each party has a keypair: a
    \emph{private} and a \emph{public} key. The public key can be safely
    shared with everyone, while private key (ideally) never leaves the
    machine.

    For encryption, Alice can \emph{encrypt} a message $m$ for Bob with his
    public key $pk_{\mathit{Bob}}$ using an encryption algorithm: $c \gets
    \mathit{Enc}(pk_{\mathit{Bob}}, m)$. She can then send $c$ to Bob, who can
    \emph{decrypt} it with his private key $sk_{\mathit{Bob}}$ using the
    respective (deterministic) decryption algorithm: $m \gets
    \mathit{Dec}(sk_{\mathit{Bob}}, c)$. It turns out that defining properly the
    desired security of such an encryption scheme is a big can of worms. Luckily
    smart people thought hard on this and came up with
    \emph{IND-CCA}\footnote{indistinguishability against chosen ciphertext
    attack} security.

    Regarding signatures, Alice can \emph{sign} a message $m$ with her
    private key (a.k.a signing key) like so: $\sigma \gets
    \mathit{Sign}(sk_{\mathit{Alice}}, m)$. She can then send the signature
    along with the message to anyone and they can then \emph{verify} that
    she indeed signed the message using her public key: $0/1 \gets
    \mathit{Verify}(pk_{\mathit{Alice}}, \sigma, m)$. The digital signature is
    used primarily for \emph{authentication}, i.e. to prove to one party
    that another party is the one which it claims it is. Analogously to
    encryption, the desired security of digital signatures is
    \emph{sEUF-CMA}\footnote{strong existential unforgeability against chosen
    message attack}.

  \subsection{Practice}
    Rather surprisingly, the API of both these primitives is simpler than that
    of symmetric encryption. The main reason is because here we don't have to
    think in terms of blocks. Let's first take a look at how key generation,
    encryption and decryption works. The asymmetric encryption standard we
    recommend is PKCS1 OAEP and the key length is 4096.

    \getenv[\oaepusage]{PKCS1_OAEP_USAGE}
    \codelisting{\oaepusage}
    Here's what you should get as output:
    \outputlisting{\oaepusage}

    Quite a long ciphertext, no? This is a minor hint as to why we still use
    symmetric encryption: It is way faster (several orders of magnitude) than
    asymmetric encryption. Many practical systems that need the asymmetric parts
    for authentication only (e.g. TLS) use a hybrid solution: they authenticate
    and negotiate a symmetric key using asymmetric cryptography and then
    transfer the actual data with the negotiated symmetric key. A little binary
    search will also reveal that the maximum plaintext size is around 470 bytes.
    Well, that's quite small. Why didn't people bother to implement modes of
    operation to allow arbitrary lengths? Precisely because asymmetric
    encryption users only tend to encrypt small chunks (except for PGP) and
    only use symmetric encryption for longer plaintexts.

    That's it for encryption, let's now see signatures. I promise they're also
    quite easy. A common, simple trick is to hash the message and sign the
    result instead of signing the message directly. This helps speed up signing
    and verifying while allowing for signing messages of practically arbitrary
    size without compromising security. We'll use SHA3 for a change. The
    recommended scheme for signatures is PKCS1 PSS (a.k.a. RSASSA-PSS), again
    with a keysize of 4096 bytes.

    We'll also seize the opportunity to see how to send public keys to other
    parties. Mind you, using files (as we do here) is only one of the many
    possible ways, and possibly the least flexible, but it shows us a simple way
    to use the \texttt{pycryptodome} API to export and import keys.

    This is Alice's script:

    \getenv[\pssusagealice]{PKCS1_PSS_USAGE_ALICE}
    \codelisting{\pssusagealice}

    And this is the code run by Bob:

    \getenv[\pssusagebob]{PKCS1_PSS_USAGE_BOB}
    \codelisting{\pssusagebob}

    Bob should get this output:
    \outputlisting{\pssusagebob}

    \begin{exercise}
      Produce the following three scripts:
      \begin{itemize}
        \item Bob generates a keypair and stores the public part to a file for
        Alice and the entire pair to another file which (we pretend that) is
        accessible only by Bob.
        \item Alice generates her keypair, encrypts and signs with it a message
        of her (i.e. your) choice. She then stores the public key, the
        ciphertext and the signature to separate files. She hashes the
        ciphertext before signing it.
        \item Bob retrieves his keypair and the three pieces of data created by
        Alice. He verifies the signature on the ciphertext, decrypts the
        message and prints it.
      \end{itemize}
    \end{exercise}

    \ifthenelse{\equal{\withsolutions}{true}}{
      \getenv[\bobgen]{GEN_KEY_BOB}
      \getenv[\alicestore]{STORE_DATA_ALICE}
      \getenv[\bobverify]{VERIFY_BOB}
      \begin{tcolorbox}[title=Solution,colback=gray!40!white]
        \codelisting{\bobgen}
        \codelisting{\alicestore}
      \end{tcolorbox}
      \begin{tcolorbox}[title=Solution (continued),colback=gray!40!white]
        \codelisting{\bobverify}
        \outputlisting{\bobverify}
      \end{tcolorbox}
    }{}

    \getenv[\quotecipher]{QUOTE_CIPHER}
    \begin{exercise*}
      Decrypt \topythonbytes{\quotecipher} using PKCS1 OAEP. The private key is
      the plaintext answer of Exercise~\ref{ex:aes:counter}.
    \end{exercise*}

    \solution{DECRYPT_PK}
\end{document}
